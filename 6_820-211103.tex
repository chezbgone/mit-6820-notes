\documentclass[class=scrartcl]{standalone}
\usepackage{chez}

\begin{document}
\section{Modeling the Heap}
Suppose we want to model the heap for separation logic.

\subsection{Heap as array}
One strategy we may use is to model the heap as an array.
If we have a C-style language with new expressions
\(\pname{malloc}(n)\) and \(*e\),
and new statements \(*e \coloneqq e\),
then we can think of the heap as a large array.

\begin{multicols}{2}
\begin{minted}{c}
x = malloc(2);
*x = 4;
*(x+1) = z;
y = *(x) + *(x+1);
{y == 4 + z}
\end{minted}
\columnbreak
\begin{minted}{c}
x = HEAP_PTR;
HEAP_PTR = HEAP_PTR + 2;
HEAP[x] = 4;
HEAP[x+1] = z;
y = HEAP[x] + HEAP[x+1];
{y == 4 + z}
\end{minted}
\end{multicols}

This requires no new machinery, and is very general.
We can even model deallocation by keeping an extra
array that keeps track of whether a position in the heap is allocated.
\begin{multicols}{2}
\begin{minted}{c}
x = malloc(2);
*x = 4;
*(x+1) = z;
y = *(x) + *(x+1);
free(x);
{y == 4 + z}
\end{minted}
\columnbreak
\begin{minted}{c}
x = HEAP_PTR;
LIVE[x] = true;
LIVE[x+1] = true;
SIZE[x] = 2;
HEAP_PTR = HEAP_PTR + 2;
HEAP[x] = 4;
Assert LIVE[x];
HEAP[x+1] = z;
Assert LIVE[x+1];
y = HEAP[x] + HEAP[x+1];
Assert LIVE[x] && LIVE[x+1];
Assert LIVE[x];
for (i=0; i<size[x]; ++i) {
  Assert LIVE[x+i];
  LIVE[x+i] = false;
}
{y == 4 + z}
\end{minted}
\end{multicols}

If we want to verify code that manipulates the heap,
like the following function that sums the value in a linked list:
\begin{minted}{c}
t = 0;
while (x != null) {
  t = t + *x;
  x = *(x+1);
}
\end{minted}
Here \(x\) points to a list of the form
\(\pname{List}\braces{{\pname{val}} \of \pint, {\pname{next}} \of \pname{List}}\).
It turns out to be extremely hard
to come up with an invariant for the loop.
It turns out that this method is not entirely useless,
as we can perform loop unrolling.
However, doing so sacrifices soundness.

\subsection{Heap separation logic}
In this method, the key idea is to break the heap into disjoint pieces,
allowing us to focus on individual pieces of the heap at a time.

\TODO[describe language]

Then, heaps are described by predicates in the following language:
\begin{itemize}
  \item \(\textsf{emp} \coloneqq \text{the heap is empty}\).
  \item \(x \mapsto y \coloneqq
            \text{the heap has one cell at location \(x\) storing \(y\)}\).
  \item \(A * B \coloneqq
            \text{the heap can be partitioned into disjoint regions,
                  one where \(A\) is true, and one where \(B\) is true}\).
\end{itemize}

\begin{example}
  If we have \((x \mapsto y) * (y \mapsto z)\),
  then we know that \(x\) maps to \(y\) and \(y\) maps to \(z\).
  Moreover, we have the implicit assumption
  that \(x \neq y\) since the regions are disjoint.
\end{example}

More formally,
\TODO[copy predicates]
\begin{align*}
  s, h \vDash B
    &\iff \bbrackets{B}s = \ptrue \\
  s, h \vDash E \mapsto F
    &\iff \set{\bbrackets{E}s} = \operatorname{dom}(h)
            \land h(\bbrackets{E}s) = \bbrackets{F}s \\
  s, h \vDash \pfalse
    &\iff \text{never} \\
  s, h \vDash P \Rightarrow Q
    &\iff ((s, h \vdash P) \Rightarrow (s, h \vdash Q)) \\
  s, h \vDash \forall x.\, P
    &\iff (\forall v \in \ZZ.\, [s \mid x \mapsto v], h \vDash P) \\
  s, h \vDash \textsf{emp}
    &\iff h = [] \\
  s, h \vDash P * Q
    &\iff \exists h_0 h_1.\,
                    (h_0 \# h_1)
                      \land (h_0 * h_1 = h)
                      \land (s, h_0 \vDash P)
                      \land (s, h_1 \vDash Q).
\end{align*}



\end{document}
