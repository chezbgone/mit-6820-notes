\documentclass[class=scrartcl]{standalone}
\usepackage{chez}

\begin{document}
\chapter{Program Logics}
Consider the following program:
\begin{minted}{c}
if (x > y) {
  t = x - y;
  while (t > 0) {
    x = x - 1;
    y = y + 1;
    t = t - 1;
  }
}
\end{minted}
We claim that for all \(x\) and \(y\),
\begin{itemize}[nosep]
  \item the loop will terminate, and
  \item if \(x > y\), at the end, \(x\) and \(y\) will be swapped
        (i.e.\ \(y\) will be the larger of the two).
\end{itemize}
It turns out that the tools we have seen so far are insufficient to prove this.

Operational semantics gave us a way to prove that
a given input will produce a given output,
or to prove that all constructs of a language in a language
will preserve some property.
However, it does not give us a way of proving
general properties about all inputs.

Type-based reasoning allows us to
design custom checkers to prove specific properties,
which is good at reasoning about values that certain variables have.
However, in this example, we want to know something
about the state of the program at a certain point,
which may not be true at other points,
which type-based reasoning is not good at.

\section{Axiomatic semantics}
We introduce a system for proving properties of programs.
The key idea is to define the semantics of constructs
by describing its effects on assertions about the program state.
This has two components:
\begin{itemize}[nosep]
  \item a language for stating assertions
  \begin{itemize}[nosep]
    \item This can be first order logic,
          or a specialized logic such as separation logic.
    \item There have been many specialized languages developed
          over the years, e.g.\ Z, Larch, JML, Spec\#.
  \end{itemize}
  \item deductive rules for establishing the truth of such assertions
\end{itemize}

\subsection{History}
In the early years, there was unbridled optimism,
where proving software was heavily endorsed by Hoare and Dijkstra.
In particular, if you can prove programs to be correct,
then bugs will be a thing of the past.

In the middle ages, in 1979, a paper by DeMillo, Lipton, and Perllis
that stated that proofs in math only work because
there is a social process in place to get people to argue about them
and internalize them.
On the other hand, proofs about programs are too boring
for social processes to form around them,
and also programs change too fast and proofs were too brittle.

Then there was a renaissance in the 2000s,
where a new generation of automated reasoning tools like Coq came out,
so that proofs can be checked automatically,
rather than by other people.
There were also many other success stories,
like Microsoft's driver verification tool.

\subsection{Hoare Triples}
A Hoare triple \(\hoare{A}{\textsf{stmt}}{B}\) is
a precondition \(A\),
a statement \(\textsf{stmt}\), and
a postcondition \(A\).
That asserts that if the precondition \(A\) holds before \textsf{stmt},
and \textsf{stmt} terminates, then the postcondition \(B\) will hold afterward.
This is a partial correctness assertion.
We may write \(\thoare{A}{\textsf{stmt}}{B}\)
to denote that \textsf{stmt} terminates.
Note that both of these are undecidable.
% undecidability is for the weak

We will use Winskel's IMP:
\begin{align*}
  e &\coloneqq n
          \mid x
          \mid e_1 + e_2
          \mid e_1 = e_2
          \mid \ptrue
          \mid \pfalse \\
  c &\coloneqq (x \coloneqq e)
          \mid c_1; c_2
          \mid \pifthenelse{e}{c_1}{c_2}
          \mid \pwhiledo{e}{c}
          \mid \pskip
\end{align*}

We introduce a language of assertions
\[
  A \coloneqq \ptrue
         \mid \pfalse
         \mid e_1 = e_2
         \mid e_1 \geq e_2
         \mid 
\]

\TODO[intro grammars]

Then partial correctness \(\hoare{A}{c}{B}\) can be defined as
\[
  \forall \sigma \, \forall \sigma'
    (\sigma \vDash A \land \angles{c, \sigma} \to \sigma') \Rightarrow
    \sigma' \vDash B
\]
Note that this is partial correctness because
termination is in the preconditions of the implication.

The power of axiomatic semantics is the ability to reach the validity
of Hoare triples by using deduction rules.

\begin{gather*}
  \prfaxiom{\vdash \hoare{A[x \to e]}{x \coloneqq e}{A}} \qquad
  \prftree{\vdash \hoare{A \land b}{c_1}{B}}
          {\vdash \hoare{A \land \pnot b}{c_2}{B}}
          {\vdash \hoare{A}{\pifthenelse{b}{c_1}{c_2}}{B}} \\
  \prftree{\vdash \hoare{A \land b}{c}{A}}
          {\vdash \hoare{A}{\pwhiledo{b}{c}}{A \land \pnot b}} \qquad
  \prftree{\vdash \hoare{A}{c_1}{C}}
          {\vdash \hoare{C}{c_2}{B}}
          {\vdash \hoare{A}{c_1; c_2}{B}} \\
  \prftree{\vdash A' \Rightarrow A}
          {\vdash \hoare{A}{c}{B}}
          {\vdash B \Rightarrow B'}
          {\vdash \hoare{A'}{c}{B'}}
\end{gather*}
The last one is called the \vocab{rule of consequence}.
\begin{itemize}
  \item The first derivation rule states that if
        we have an assertion \(A\) where we substitute in \(e\) for \(x\),
        then running \(x \coloneqq e\) will make the assertion \(A\) hold.
        For instance, we have
        \[
          \prfaxiom{\vdash \hoare{x + 1 > 0}{x \coloneq x + 1}{x > 0}}
        \]
  \item The second rule states that
\end{itemize}


\subsection{Soundness and completeness}
Now that we have the rules, we want to prove that they are sound and complete.
\begin{itemize}[nosep]
  \item Soundness means that we will never be able to prove
        something that is not true.
  \item Completeness means that if a statement is true,
        then we can deductively prove it.

\end{itemize}












\end{document}
