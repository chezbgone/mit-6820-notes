\documentclass[class=scrartcl]{standalone}
\usepackage{chez}

\begin{document}
\chapter{Types for imperative programs}
Recall that when evaluating lambda calculus,
there is no state to keep track of.
We had the following inductive definition
\[
  \prfaxiom{x \to x} \qquad
  \prftree{e \to e'}
          {\plambda{x}{e} \to \plambda{x}{e'}} \qquad
  \prftree{e_1 \to \plambda{x}{e_1'}}
          {e_1'[e_2/x] \to e_3}
          {e_1 e_2 \to e_3}
\]

We want to do the same thing for imperative programs.
Since we now have state, the configuration must now include it,
rather than just the types of each variable.

We will be using the language IMP,
which is the imperative equivalent to lambda calculus
in the sense that it is the simplest imperative language
that we can add things to.

\section{Introduction}
The grammar of IMP is:
\begin{gather*}
  e \coloneqq n
         \mid x
         \mid e_1 + e_2
         \mid e_1 == e_2
         \mid \ptrue
         \mid \pfalse \\
  c \coloneqq (x \coloneqq e)
         \mid c_1; c_2
         \mid \pifthenelse{e}{c_1}{c_2}
         \mid \pwhiledo{e}{c}
         \mid \pskip
\end{gather*}
where \(e\) represents expressions and \(c\) represents commands.


\subsection{Big step semantics}
Since there is a distinction between expressions and commands,
we now need two types of judgements:
\begin{itemize}[nosep]
  \item expressions that result in a value \(\angles{e, \sigma} \to n\)
  \item commands that change the state \(\angles{c, \sigma} \to \sigma'\)
\end{itemize}
Here, the state \(\sigma\) is a mapping from variables to values.

The rules for expressions are similar to what we had before:
\[
  \prfaxiom{\angles{N, \sigma} \to n} \qquad
  \prftree{\angles{e_1, \sigma} \to n_1}
          {\angles{e_2, \sigma} \to n_2}
          {n = n_1 + n_2}
          {\angles{e_2 + e_2, \sigma} \to n} \qquad
  \prfaxiom{\angles{x, \sigma} \to \sigma(x)}
\]
where the third rule above is to read from variables.
For commands, the state is changed:
\begin{gather*}
  \prftree{\angles{e, \sigma} \to e'}
          {\angles{X \coloneqq e, \sigma} \to \sigma[X \to e']} \qquad
  \prftree{\angles{c_1, \sigma} \to \sigma''}
          {\angles{c_2, \sigma''} \to \sigma'}
          {\angles{c_1 ; c_2, \sigma} \to \sigma'} \\
  \prftree{\angles{e, \sigma} \to \ptrue}
          {\angles{c_1, \sigma} \to \sigma'}
          {\angles{\pifthenelse{e}{c_1}{c_2}, \sigma} \to \sigma'} \qquad
  \prftree{\angles{e, \sigma} \to \pfalse}
          {\angles{c_2, \sigma} \to \sigma'}
          {\angles{\pifthenelse{e}{c_1}{c_2}, \sigma} \to \sigma'} \\
  \prftree{\angles{e, \sigma} \to \pfalse}
          {\angles{c_2, \sigma} \to \sigma}
          {\angles{\pwhiledo{e}{c}, \sigma} \to \sigma'} \qquad
  \prftree{\angles{e, \sigma} \to \ptrue}
          {\angles{c, \sigma} \to \sigma''}
          {\angles{\pwhiledo{e}{c}, \sigma''} \to \sigma'}
          {\angles{\pwhiledo{e}{c}, \sigma} \to \sigma'}
\end{gather*}
For the \(\pwhiledo{\ptrue}{c}\) case, note that the last two premises
is equivalent to the sequential composition rule.
Therefore, the last rule is equivalent to
\[
  \prftree{\angles{e, \sigma} \to \ptrue}
          {\angles{c; \pwhiledo{e}{c}, \sigma} \to \sigma'}
          {\angles{\pwhiledo{e}{c}, \sigma} \to \sigma'}
\]

\subsection{Small step semantics}
Recall that a redex is some rule \(r \to v\),
and a context is an expression with some hole \(C = \bullet \mid \cdots\).
Since we now have state, our local reductions must be of the form
\(\angles{r, \sigma} \to \angles{v, \sigma'}\).
Our reductions are fairly intuitive:
\begin{itemize}
  \item \(\angles{x, \sigma[x = n]} \to \angles{n, \sigma[x = n]}\)
  \item \(\angles{n_1 + n_2, \sigma} \to \angles{n, \sigma}\)
        where \(n = n_1 + n_2\).
  \item \(\angles{x \coloneqq n, \sigma} \to \angles{\pskip, \sigma[x \gets n]}\)
  \item \(\angles{\pskip; c, \sigma} \to \angles{c, \sigma}\)
  \item \(\angles{\pifthenelse{\ptrue}{c_1}{c_2}, \sigma}
          \to \angles{c_1, \sigma}\)
  \item \(\angles{\pifthenelse{\ptrue}{c_1}{c_2}, \sigma}
          \to \angles{c_2, \sigma}\)
  \item \(\angles{\pwhiledo{b}{c}, \sigma}
          \to \angles{\pifthenelse{b}{c; \pwhiledo{b}{c}}{\pskip}, \sigma}\)
\end{itemize}
To do global reduction, we repeatedly identify a redex
(i.e.\ write the program as \(C[r]\) for some \(C\) and \(r\)), and reduce it.
To do this, we need to identify the next redex,
which defines the evaluation order.
We define
\[
  H \coloneqq \bullet
         \mid n + H
         \mid H + e
         \mid (x \coloneqq H)
         \mid \pifthenelse{H}{c_1}{c_2}
         \mid H; c
\]
Note that in \(e_1 + e_2\), \(e_1\) is evaluated before \(e_2\).
In general, redexes and contexts define
evaluation order and short circuit behavior.

\begin{lemma}[Decomposition]
  If \(c\) is not \(\pskip\), then there exists a unique \(H\) and \(r\)
  such that \(c = H[r]\).
\end{lemma}
This is similar to the progress lemma we had before,
but this time, we can guarantee determinism
with the uniqueness of \(H\) and \(r\).


\subsection{References}
Many imperative programs have a heap,
which allows us to maintain unbounded state.
We will talk about one approach to the heap,
which comes from the ML world.
One thing that we have in OCaml is an explicit heap.
In particular, we can create heap allocated objects and references them.
However, it still tries to preserve much of what we like
about functional languages.

To add references into a \emph{functional} language,
it helps to think about the type also.
We modify lambda calculus like so:
\begin{align*}
  \tau &\coloneqq \cdots \mid \tau \pref \\
  e    &\coloneqq \cdots \mid \pref e
                         \mid (e_1 \coloneqq e_2)
                         \mid e_1 ; e_2
                         \mid {!e}
\end{align*}
The idea is that \(\pref e\) creates a new ref with initial value \(e\),
and \(!e\) extracts the value from the ref.

To model the heap, we have the following definition:
\[
  h \coloneqq \nullset \mid h, a \to (v \of \tau)
\]
Then, a program is a heap along with an expression,
and heap addresses act like bound variables in expressions:
\[
  p = \pheapin{h}{e}
\]
The new contexts added to support the heap are:
\[
  H = \dots \mid \pref H
            \mid (H \coloneqq e)
            \mid (\textsf{addrs} \coloneqq H)
            \mid {!H}
\]
Since the heap is a global object, there are no local reduction rules.
We do, however, have new global reduction rules:
\begin{itemize}[nosep]
  \item \(\pheapin{h}{H[\pref v \of \tau]}
          \to \pheapin{(h[a \to v \of \tau])}{H[a]}\)
  \item \(\pheapin{h}{H[!a]} \to \pheapin{h}{H[v]}\)
        where \(a \to v \in h\)
  \item \(\pheapin{h}{H[a \coloneqq v]}
          \to \pheapin{(h[a \to v] \of \tau)}{H[\punit]}\)
\end{itemize}
The typing rules for each of the \(\pref\) primitives are
\[
  \prftree{\Gamma \vdash e \of \tau}
          {\Gamma \vdash (\pref e \of \tau) \of \tau \pref} \qquad
  \prftree{\Gamma \vdash e \of \tau \pref}
          {\Gamma \vdash {!e} \of \tau} \qquad
  \prftree{\Gamma \vdash e_1 \of \tau \pref}
          {\Gamma \vdash e_2 \of \tau}
          {\Gamma \vdash e_1 \coloneqq e_2 \of \punit}
\]

One problem with this is how they interact with polymorphism.
Consider the following:
\begin{align*}
  \pname{let}{} & x \of {\forall t. ((t \to t) \pref)}
                    = \Lambda t. \pref (\plambda{x \of t}{x}) \\[-1ex]
  \pname{in}{}  & \begin{lgathered}[t]
                    x[\pbool] \coloneqq \plambda{x \of \pbool}{\pnot x} \\[-1ex]
                    (!x[\pint])5
                  \end{lgathered}
\end{align*}
The solution is to disallow side effects in \(\pname{let}\).

\subsection{Typing rules}
For imperative programs, the typing rules are not too interesting:
\begin{gather*}
  \prftree{x \of \tau \in \Gamma}
          {\Gamma \vdash x \of \tau} \qquad
  \prftree{\Gamma \vdash e_1 \of \pint}
          {\Gamma \vdash e_2 \of \pint}
          {\Gamma \vdash e_1 + e_2 \of \pint} \qquad
  \prftree{\Gamma \vdash N \of \pint} \qquad
  \prftree{\Gamma \vdash e \of \pbool}
          {\Gamma \vdash c_1 \of \punit}
          {\Gamma \vdash c_2 \of \punit}
          {\Gamma \vdash \pifthenelse{e}{c_1}{c_2} \of \punit} \\
  \prftree{\Gamma \vdash e \of \tau}
          {\Gamma \vdash x \of \tau}
          {\Gamma \vdash x \coloneqq e \of \punit} \qquad
  \prftree{\Gamma \vdash e \of \pbool}
          {\Gamma \vdash c \of \punit}
          {\Gamma \vdash \pwhiledo{e}{c} \of \punit}
\end{gather*}

To prove progress and preservation,
it it similar to what we did for functional programs.
The difference is that we need to take into account the mutable state.
More specifically, we have
\begin{description}
  \item [Preservation] If the mutable state is well typed,
        then after evaluating one step it will remain well typed.
  \item [Progress] If the mutable state is well typed,
        then we can make progress.
\end{description}







\end{document}
