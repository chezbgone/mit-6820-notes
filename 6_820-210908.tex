\documentclass[class=scrartcl]{standalone}
\usepackage{chez}

\begin{document}
\chapter{Introduction}
The goal of this course is to give an introduction to
the ideas in the programming language community.
In particular, we want to understand the tools that help us
think about a program's behavior.
This has applications in
\begin{itemize}
  \item finding bugs,
  \item designing languages to prevent bugs,
  \item program synthesis, and
  \item program optimization.
\end{itemize}
As a prerequisite,
we must be able to understand the program behavior ourselves.
In particular, we want to learn
  how to define programs and languages unambiguously,
  how to prove theorems about the behavior of programs, and
  how to automate this process.

The big ideas in this class include:
\begin{description}
  \item[Operational semantics] which gives programs meaning
    via interpreters that have the added feature that we can
    make mathematical statements while interpreting.
  \item[Program proofs as inductive invariants]
    where we take advantage of the structure
    that programs maintain within a loop
  \item[Abstraction] where we model programs with specifications,
    as opposed to the implementation.
    E.g.\ considering a program as a bag of statements without order.
  \item[Modularity] where we split programs up into pieces to analyze
    smaller parts separately.
\end{description}

For this class, we will be using
  Haskell,
  Coq,
  OCaml, and
  the Spin model checker.


\chapter{Functional Programming}
\section{Introduction}
In functional programming, functions are first-class values.
This means that, for instance, you can pass functions as values and
  return functions in other functions.
Another (more defining) feature of functional programming languages are that
they are based on lambda calculus.

The following is a function definition
\begin{minted}{haskell}
   increment x = x + 1
-- [1]      [2]  [ 3 ]
\end{minted}
\begin{enumerate}
  \item \mintinline{haskell}{increment} is the name of
        the function we are defining.
  \item The first \mintinline{haskell}{x} is the parameter of the function.
  \item The \mintinline{haskell}{x + 1} is the body of the function.
        It is how the function is evaluated if an argument is supplied.
\end{enumerate}

\subsection{Scoping}
When we want to introduce components of expressions,
we can use the \mintinline{haskell}{let ... in ...} construction: % chktex 26 chktex 11 chktex 12
\begin{minted}{haskell}
let dx = 5 - 2
    dy = 3 - 1
 in dx * dx + dy * dy
\end{minted}
which is equivalent to
\mintinline{haskell}{(5 - 2) * (5 - 2) + (3 - 1) * (3 - 1)}. % chktex 8
The order of the statements inside of the \mintinline{haskell}{let}
do not matter, even if one depends on the other.

When we have nested \mintinline{haskell}{let ... in ...} expressions % chktex 26 chktex 11 chktex 12
with the same variable names, the value of the variables are based on
the innermost scope.
For instance,
\begin{minted}{haskell}
let x = 3
    y = 4
 in let y = 5
     in x + y
\end{minted}
and
\begin{minted}{haskell}
let x = 3
    y = 4
 in x + 5
\end{minted}
are equivalent.

While the compiler may understand what this program means,
and the program only has one interpretation,
it is confusing for humans to understand it when different values
are bound to the same name.
To solve this issue, we can rename some variables
(formally called \(\alpha\)-renaming).
This gives us the equivalent code of
\begin{minted}{haskell}
let x = 3
    y = 4
 in let y' = 5
     in x + y'
\end{minted}
Note that the apostrophe (\mintinline{haskell}{'}, typically called ``prime'')
is indeed a valid character to use in variable names.

However, we should be careful when doing this, to make sure we are not
overriding another value.
In particular, we do not want to rename \mintinline{haskell}{y}
to \mintinline{haskell}{x}, as this changes the behavior of the program.
\begin{minted}{haskell}
let x = 3
    y = 4
 in let x = 5
     in x + x
\end{minted}
Of course, the easy solution is to just not code like this in the first place.

\subsection{Currying}
In Haskell, a function can only have one parameter.
However, we can still define functions that
seem to take in multiple parameters.
\begin{minted}{haskell}
let plus x y = x + y
 in plus 3 4
-- 7
\end{minted}
The way we interpret this is \mintinline{haskell}{(plus x) y = x + y}.
In particular, \mintinline{haskell}{plus} is a function that takes in
a number \mintinline{haskell}{x} and returns another function
\mintinline{haskell}{plus x}.
This new function called \mintinline{haskell}{plus x} then takes in
another number \mintinline{haskell}{y} and adds \mintinline{haskell}{x} to it.
This idea of transforming a function that takes multiple arguments
into one that takes in arguments one at a time is called \vocab{currying}.
This is useful because it lets us partially apply functions:
\begin{minted}{haskell}
let plus x y = x + y
    increment = plus 1
 in increment 3
-- 4
\end{minted}
In general, we interpret \mintinline{haskell}{(a b c d)} as
\mintinline{haskell}{(((a b) c) d)}, i.e.\ function application
associates to the left.


\subsection{Infix operators}
Note that we have been defining \mintinline{haskell}{plus x y = x + y}.
It would be convenient if we are able to say that \mintinline{haskell}{plus}
is the same thing as \mintinline{haskell}{+}.
We can indeed do this. In the expression \mintinline{haskell}{(x + y)},
we are using \mintinline{haskell}{+} as an infix operator,
but we can convert it to a normal function by
writing it as \mintinline{haskell}{(+)}.
In particular, our original definition is equivalent too
\mintinline{haskell}{plus x y = (+) x y}.
We can simplify this further to the code \mintinline{haskell}{plus = (+)}.

Conversely, if we have a named function that takes in (at least) two arguments,
we can make it infix by surrounding the name in backticks, like
\mintinline{haskell}{(x `plus` y)}. % chktex 33








\end{document}
