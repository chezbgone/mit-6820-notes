\documentclass[class=scrartcl]{standalone}
\usepackage{chez}

\begin{document}
\chapter{Abstract Interpretation}
So far, we have learned:
\begin{itemize}
  \item Operational semantics, which tells us
        how programs behave on a given input.
  \item Types, which are annotations that describes the data
        that can be referred to by a variable.
        We can describe global properties execution,
        but only one variable at a time.
        Moreover, this is limited by the type system designer.
  \item Program logics, which are annotations that describe
        properties of the state at a given point in the program.
        It is easy to describe complex properties of program state,
        but messy to describe properties that hold over time.
        However, the analysis can be expensive,
        and annotations (loop invariants) are hard to infer.
\end{itemize}

\section{Motivation}
Consider the following program:
\begin{minted}{text}
{true}
y=0;
while(x<10){
  x = x+1;
  y = y+2;
}
{even(y)}
\end{minted}
To prove this, we must find a loop invariant \(A\) such that
\[
  \prftree{\vdash \hoare{A \land b}{c}{A}}
          {\vdash \hoare{A}{\pwhiledo{b}{c}}{A \land \pnot b}}
\]
The idea is that \(A\) is a set of states,
and \(c\) takes elements in \(A \land b\)
to elements in \(A\).

Consider the simpler problem
\[
  \prftree{\vdash \hoare{A}{c}{A}}
          {\vdash \hoare{A}{\pwhiledo{b}{c}}{A}}
\]
This rule is strictly weaker, and many correct programs
cannot be proved with it.
The idea of this is that \(c\) transforms states in \(A\) to states in \(A\).

Now the question is how to come up with the invariant \(A\).
We want a set \(A\) such that \(\vdash \hoare{A}{c}{A}\),
and it has to be both small enough to prove the postcondition,
and weak enough to prove the precondition.

If we let
\[
  F(P) = \operatorname{wpc}(c, P) \land \textsf{postcondition},
\]
then we want the greatest fixpoint solution of \(A = F(A)\).
Then more questions arise:
\begin{itemize}
  \item Will this converge/always find solutions.
  \item When is it better to use \(\operatorname{wpc}\) vs.\ \(\operatorname{spc}\)?
  \item How do we minimize loss of precision?
\end{itemize}

\section{Abstract Interpretation}
\subsection{Partial orders and lattices}
To talk about abstract interpretation,
we will first need to learn about partial orders.

A \vocab{partial order} is a set \(P\) with a relation \(\leq\)
such that for all \(x, y, z \in P\),
\(\leq\) is reflexive, asymmetric, and transitive.
We can then use this to define upper and lower bounds
We use \(\bigvee S\) for least upper bound of \(S \subseteq P\).
Similarly, we use \(\bigwedge S\) for greatest lower bound of \(S \subseteq P\).
Sometimes, people use \(\sqcup\), \(\sqcap\), \(\cup\), or \(\cap\).

We say that \(y\) \vocab{covers} \(x\) if \(x < y\) and
\(x \leq z < y \implies x = z\).
Intuitively, this says that nothing is between \(x\) and \(y\).

If \(x \wedge y\) and \(x \vee y\) exists for all \(x, y \in P\),
then we call \(P\) a \vocab{lattice}.
If \(\bigwedge S\) and \(\bigvee S\) exists for all \(S \subseteq P\),
then we call \(P\) a \vocab{complete lattice}.
An example of an incomplete lattice is the integers,
but we can ``complete'' it by adding the elements \(- \infty\) and \(\infty\).

The greatest element of \(P\) (if it exists) is called \vocab{top} (\(\top\)).
The least element of \(P\) (if it exists) is called \vocab{bottom} (\(\bot\)).

A set \(S \subseteq P\) is a \vocab{chain} if for all \(x, y \in S\),
either \(y \leq x\) or \(x \leq y\).

Given two lattices \(L\) and \(Q\),
we can construct the \vocab{product lattice} easily:
\[
  (l_1, q_1) \sqsubseteq (l_2, q_2) \iff
    l_1 \sqsubseteq l_2 \land q_1 \sqsubseteq q_2.
\]
This construction easily extends to an arbitrary number of lattices.

\subsection{Lattice of predicates}
Recall our program
\begin{minted}{text}
{true}
y=0;
while(x<10){
  x = x+1;
  y = y+2;
}
{even(y)}
\end{minted}
Consider the predicates \(\set{\top, \textsf{even}, \textsf{odd}, \bot}\),
where \(\top\) means that the argument is even \emph{or} odd.
Then this forms a lattice, and we can take the product lattice with itself:
\(\angles{x = \textsf{even}, y = \textsf{odd}} \sqsubseteq
    \angles{x = \top, y = \textsf{odd}}\).

Now we can find a greatest fixpoint solution as long as
\(F(P) = \operatorname{wpc}(c, P) \land \textsf{postcondition}\),
is monotonic in our lattice,
i.e.\ \(x \sqsubseteq y \implies F(x) \sqsubseteq F(y)\).

For instance, let \(x_\bot\) be the least fixed point of \(f \colon L \to L\).
Then let \(x_0 = \bot\) and \(x_i = f(x_{i-1})\).
By induction, \(x_i \sqsubseteq x_\bot\).
Moreover, the chain \(x_i\) is ascending.
If \(L\) has no infinite ascending chains, then eventually \(x_i = x_\bot\).

We can do the same rick with the greatest fixed point
if we start with \(x_0 = \top\).

In our problem, we have
\begin{align*}
  F(P) &= \operatorname{wpc}(c, P) \land \textsf{postcondition} \\
  P_0  &= \angles{x = \top, y = \top}                           \\
  P_1  &= \angles{x = \top, y = \textsf{even}}                  \\
  P_2  &= \angles{x = \top, y = \textsf{even}},
\end{align*}
which gives us a fixed point, which is a valid loop invariant.

\begin{theorem}[Knaster-Tarski]
  Let \(L\) be complete and
  \(f \colon L \to L\) be an order preserving function.
  Then the set of fixed points of \(f\) in \(L\) is also a complete lattice.
\end{theorem}

For if statements, we have a similar relaxed rule
\[
  \prftree
    {\vdash \hoare{A \land b}{c_1}{B}}
    {\vdash \hoare{A \land \pnot b}{c_2}{B}}
    {\vdash \hoare{A}{\pifthenelse{b}{c_1}{c_2}}{B}}
  \implies
  \prftree
    {\vdash \hoare{A}{c_1}{B}}
    {\vdash \hoare{A}{c_2}{B}}
    {\vdash \hoare{A}{\pifthenelse{b}{c_1}{c_2}}{B}}
\]

\TODO[]








\end{document}
