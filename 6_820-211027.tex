\documentclass[class=scrartcl]{standalone}
\usepackage{chez}

\begin{document}



\section{Verification}
The Hoare triple we want to prove is
\begin{minted}[highlightlines={1,10}]{c}
{x=x0, y=y0}
if (x > y) {
  t = x - y;
  while (t > 0) {
    x = x - 1;
    y = y + 1;
    t = t - 1;
  }
}
{x0 > y0 => y=x0 and x=y0}
\end{minted}
To prove this, we have the rule
\[
  \prftree{\hoare{x = x_0 \land y = y_0 \land x > y}
                 {P}
                 {x_0 > y_0 \Rightarrow y = x_0 \land x = y_0}
          }
          {\hoare{x = x_0 \land y = y_0 \land x \leq y}
                 {\pskip}
                 {x_0 > y_0 \Rightarrow y = x_0 \land x = y_0}
          }
          {\hoare{x = x_0 \land y = y_0}
                 {\pifthenelse{x > y}{P}{\pskip}}
                 {x_0 > y_0 \Rightarrow y = x_0 \land x = y_0}}
\]
The right premise is proved by using the rule
\[
  \prfaxiom{\hoare{A}{\pskip}{A}}
\]
To prove the left, we have the rule
\[
  \prftree[r]{\scriptsize (A)}
          {\hoare{x = x_0 \land y = y_0 \land x > y}
                 {t = x - y}
                 {C}}
          {\hoare{C}
                 {\pwhiledo{t > 0}{P_2}}
                 {x_0 > y_0 \Rightarrow y = x_0 \land x = y_0}}
          {\hoare{x = x_0 \land y = y_0 \land x > y}
                 {t = x - y;
                  \pwhiledo{t > 0}{P_2}}
                 {x_0 > y_0 \Rightarrow y = x_0 \land x = y_0}}
\]
The big question in program verification is what \(C\) should be.
If we think about it a bit, we may notice that we have throughout the loop,
we have \(x_0 = y + t\) and \(y_0 = x - t\).
Therefore, we choose \(C = (t \geq 0 \land y + t = x_0 \land x - t = y_0)\).
Therefore, to deduct the left of (A), we have
\[
  \prftree
  {\prfaxiom{
     \hoare{x - y \geq 0 \land y + (x - y) = x_0 \land x - (x - y) = y_0}
           {t = x - y}
           {t \geq 0 \land y + t = x_0 \land x - t = y_0}
     }
  }
  {\hoare{x = x_0 \land y = y_0 \land x > y}
                 {t = x - y}
                 {x_0 > y_0 \Rightarrow y = x_0 \land x = y_0}}
\]
where the first line is from the assignment rule,
and the second line is from the rule of consequence.

A more general strategy is to construct a \vocab{rank function} \(F\)
that has these properties:
\begin{itemize}[nosep]
  \item <++>
\end{itemize}
In our example, we would choose \(F = t\).


To mechanize this strategy, we will use the idea of weakest preconditions.
Note that we still need to manually select the invariant.
We define the \vocab{weakest precondition} \(\wpc(c, A) = P\)
to be the weakest predicate \(P\) such that \(\vdash \hoare{P}{c}{A}\).
The idea is \TODO[]
We have
\begin{align*}
  \wpc(\pskip, Q) &= Q \\
  \wpc(x = e, Q) &= Q[x \to e] \\
  \wpc(c_1; c_2, Q) &= \wpc(c_1, \wpc(c_2, Q)) \\
  \wpc(\pifthenelse{b}{c_1}{c_2}, Q)
    &= (b \lor \wpc(c_1, Q))
       \land (\pnot b \lor \wpc(c_2, Q))
\end{align*}
However, when we want to work with loops, this does not work well.
We can get around this by using a \vocab{verification condition},
which is stronger than the weakest precondition.
In particular, we define
\[
  \operatorname{VC}(\pwhiledo{e}{c}, B)
    = I \land \forall x_1 \dots x_n. I
        \Rightarrow (e \Rightarrow \operatorname{VC}(c, I)
                       \land \pnot e \Rightarrow B),
\]
where \(x_i\) are the variables modified in \(c\).









\end{document}
