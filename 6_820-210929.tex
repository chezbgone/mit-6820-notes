\documentclass[class=scrartcl]{standalone}
\usepackage{chez}

\begin{document}
\begin{question}
  How do we formalize this process?
\end{question}

\subsection{Constraints}
First, we will formalize what a contraint actually is,
talk about how to generate them.

Consider the language of constraints:
\[
  C := \tau_1 = \tau_2 \mid C \lor C
                       \mid \exists \tau. C
\]
We can build constraints from the typing rules
as we build the type checking tree.
The base cases are
\[
  \bbrackets{\Gamma \vdash x \of \tau} = (\Gamma(x) = \tau) \qquad
  \bbrackets{\Gamma \vdash N \of \tau} = (\pint = \tau).
\]
The inductive cases are
\begin{gather*}
  \bbrackets{\Gamma \vdash e_1 e_2 \of \tau}
    = \exists a (\bbrackets{\Gamma \vdash e_1 \of a \to \tau} \land
                 \bbrackets{\Gamma \vdash e_2 \of a}) \\
  \bbrackets{\Gamma \vdash \lambda x. e \of \tau}
    = \exists a_1 a_2. (\bbrackets{\Gamma, x \of a_1 \vdash e \of a_2} \land
                         \tau = a_1 \to a_2) \\
  \bbrackets{\Gamma \vdash e_1 + e_2 \of \tau}
    = \exists a_1 a_2. (\bbrackets{\Gamma \vdash e_1 \of \pint} \land
                        \bbrackets{\Gamma \vdash e_2 \of \pint} \land
                        \tau = \pint)
\end{gather*}

\begin{example}
  Suppose we want to determine the type of
  \[
    \lambda x. x + 1.
  \]
  We can recursively evaluate the type judgement
  \begin{align*}
    \bbrackets{\Gamma \vdash (\plambda{x}{x+1}) 2 \of \tau}
      &= \exists a_1.\,
        \textcolor{chezblue}{
          \bbrackets{\Gamma \vdash (\plambda{x}{x+1}) \of a_1 \to \tau}} \land
        \textcolor{chezgreen}{\bbrackets{\Gamma \vdash 2 \of a_1}} \\
      &= \exists a_1.\,
        \textcolor{chezblue}{(\exists a_2 a_3.\,
          \textcolor{chezpurple}{
            \bbrackets{\Gamma, x \of a_2 \vdash x + 1 \of a_3}} \land
          (a_1 \to \tau = a_2 \to a_3)
        )} \land
        \textcolor{chezgreen}{(a_1 = \pint)} \\
      &= \exists a_1.\,
        \textcolor{chezblue}{(\exists a_2 a_3.\,
          \textcolor{chezpurple}{
            \bbrackets{\Gamma, x \of a_2 \vdash x \of \pint} \land
            \bbrackets{\Gamma, x \of a_2 \vdash 1 \of \pint} \land
            (a_3 = \pint)
          } \land
          (a_1 \to \tau = a_2 \to a_3)
        )} \land
        \textcolor{chezgreen}{(a_1 = \pint)} \\
      &= \exists a_1.\,
        \textcolor{chezblue}{(\exists a_2 a_3.\,
          \textcolor{chezpurple}{
            (a_2 = \pint) \land
            (\pint = \pint) \land
            (a_3 = \pint)
          } \land
          (a_1 \to \tau = a_2 \to a_3)
        )} \land
        \textcolor{chezgreen}{(a_1 = \pint)}.
  \end{align*}
  and this gives us an expression in the language of constraints.
\end{example}

Now that we have constraints,
we want to determine if they are equal, and
whether they can be satisfied with
an appropriate choice of type variables.

To answer the first question,
to determine when two types \(\tau_a\) and \(\tau_b\) are equal,
we use \vocab{structural equality}.
In particular, \(\tau_a = \tau_b\) when their constructors are the same,
and their contents are the same.
For instance, if \(\tau_a = \tau_1 \to \tau_2\) and
                 \(\tau_b = \tau_3 \to \tau_4\),
then \(\tau_a = \tau_b \iff \tau_1 = \tau_3 \land \tau_2 = \tau_4\).

\subsection{Unification}
The answer to the question of how to solve constraints is more complicated.
We define \vocab{unification} to be the process of determining
whether two types can be made equal by selecting appropriate
substitutions for type variables.

First we will formally define types and type substitution.
The language of \vocab{types} is
\begin{alignat*}{4}
  \tau :={} & \iota && \qquad
    \text{base types
          (\mintinline{haskell}{Int}, \mintinline{haskell}{Bool}, \dots)} \\
  \mid{} & t && \qquad\text{type variables} \\
  \mid{} & \tau_1 \to \tau_2 && \qquad\text{function types}.
\end{alignat*}
A \vocab{type substitution} is a map
\begin{align*}
  S \colon& \set{\text{type variables}} \to \set{\text{types}} \\[-1ex]
  S ={}& [\tau_1/t_1, \tau_2/t_2, \dots, \tau_n/t_n],
\end{align*}
and when \(\tau' = S \tau\),
we say \(\tau'\) is a \vocab{substitution instance} of \(\tau\).

\begin{example}
  If we have \(S_1 = [(t \to \pbool) / t_1]\) and \(S_2 = [\pint / t]\), then
  \begin{align*}
    S_2 S_1 (t_1 \to t_1) &= S_2 ((t \to \pbool) \to (t \to \pbool)) \\
      &= (\pint \to \pbool) \to (\pint \to \pbool).
  \end{align*}
\end{example}

To determine whether two types are able to be unified,
we use the following algorithm:
\begin{minted}[escapeinside=||]{haskell}
unify(|\(\tau_1\)|, |\(\tau_2\)|) =
  case (|\(\tau_1\)|, |\(\tau_2\)|) of
    |\((\tau_1, t_2\))| -> |\([\tau_1/t_2]\)| given |\(t_2 \notin FV(\tau_1)\)|
    |\((t_1, \tau_2\))| -> |\([\tau_2/t_1]\)| given |\(t_1 \notin FV(\tau_2)\)|
    |\((\iota_1, \iota_2\))| -> if (eq |\(\iota_1\)| |\(\iota_2\)|) then []
                          else Fail
    |\((\tau_{11} \to \tau_{12}, \tau_{21} \to \tau_{22}\))| -> let |\(S_1 = {}\)|unify|\((\tau_{11}, \tau_{21})\)|
                              |\(S_2 = {}\)|unify|\((S_1 \tau_{12}, S_1 \tau_{22})\)|
                          in |\(S_2 S_1\)|
    otherwise -> Fail
\end{minted}

\subsection{Putting it together}
Now that we have formalized both constraints and unification,
we can combine them into a type inference algorithm.
In the examples above (e.g.\ \cref{ex:type-inference}),
we collected all of the constraints,
and then at the end we solved the system.
However in practice, this can be inefficient if your program is large,
so we often solve the constraints incrementally,
and build the substitution map incrementally.

Let us come up with an inference algorithm \(W\) such that
\(W(\textrm{TE}, e) = (S, \tau)\), where \(S(\textrm{TE}) \vdash e \of \tau\).
Here, we are writing \(\textrm{TE}\) instead of \(\Gamma\) for convenience.
Intuitively, the type environment \(\textrm{TE}\) stores
  the most general type of each identifier while
the substitution \(S\) stores the changes in type variables.

Naturally, we will do casework on \(e\). We have
\begin{gather*}
  \bbrackets{\Gamma \vdash N \of \tau} = (\pint = \tau) \qquad
    \bbrackets{\Gamma \vdash x \of \tau} = (\Gamma(x) = \tau) \\
  \bbrackets{\Gamma \vdash e_1 e_2 \of \tau}
    = \exists a (\bbrackets{\Gamma \vdash e_1 \of a \to \tau} \land
                 \bbrackets{\Gamma \vdash e_2 \of a}) \\
  \bbrackets{\Gamma \vdash \lambda x. e \of \tau}
    = \exists a_1 a_2. (\bbrackets{\Gamma, x \of a_1 \vdash e \of a_2} \land
                         \tau = a_1 \to a_2).
\end{gather*}
This gives us the algorithm
\begin{minted}[escapeinside=||]{haskell}
w(te, e) = case e of
  c    -> ({}, Int)
  x    -> if (x |\(\in\)| te) then ({}, te[x])
                      else Fail
  |\(\lambda\)|x.e -> let (|\(S_1\)|, |\(\tau_1\)|) = w(te |\(\union\)| {x:u}, e)
           in (|\(S_1\)|, |\(S_1\)|(u) -> |\(\tau_1\)|)
  |\(e_1 e_2\)|  -> let (|\(S_1\)|, |\(\tau_1\)|) = w(te, |\(e_1\)|)
              (|\(S_2\)|, |\(\tau_2\)|) = w(|\(S_1\)|(te), |\(e_2\)|)
              |\(S_3\)| = unify(|\(S_2\)|(|\(\tau_1\)|), |\(\tau_2\)| -> u)
           in (|\(S_3 S_2 S_1\)|, |\(S_3\)|(u))
\end{minted}

\begin{example}
  We can even run this algorithm by hand:
  Let's say we are trying to compute
  \(W(\nullset, (\plambda{f}{f 5})(\plambda{x}{x}))\).
  \begin{adhoctheorem*}{Computation}[Application]
    Let \((S_1, \tau_1) = W(\nullset, \plambda{f}{f 5})
                        = (S_1', S_1'(u_1) \to \tau_1')
                        = ([\pint \to u_2 / u_1], (\pint \to u_2) \to u_2)\),
    with subvariables obtained from the following subcomputation:
    \begin{adhoctheorem*}{Computation}[Lambda]
      Let \((S_1', \tau_1') = W(\nullset \union \set{f \of u_1}, f 5)
                            = (S_3'' S_2'' S_1'', S_3'' u_2)
                            = ([\pint \to u_2 / u_1], u_2)\)
      with subvariables obtained from the subcomputation:
      \begin{adhoctheorem*}{Computation}[Application]
        Let \((S_1'', \tau_1'') = W(\set{f \of u_1}, f) = (\nullset, u_1)\).

        Let \((S_2'', \tau_2'') = W(\set{f \of u_1}, 5) = (\nullset, \pint)\).

        Let \(\begin{aligned}[t]
            S_3'' &= \pname{unify}(S_2''(u_1), \pint \to u_2) \\[-1ex]
                  &= \pname{unify}(u_1, \pint \to u_2) \\[-1ex]
                  &= [\pint \to u_2 / u_1]
          \end{aligned}\).
      \end{adhoctheorem*}
    \end{adhoctheorem*}

    Let \((S_2, \tau_2) = W(S_1' \nullset, \plambda{x}{x})
                        = (S_1', S_1'(u_3) \to \tau_1')
                        = (\nullset, u_3 \to u_3)\),
    with subvariables obtained from the subcomputation:
    \begin{adhoctheorem*}{Computation}[Lambda]
      Let \((S_1', \tau_1') = W(\nullset \union \set{x \of u_3}, x)
                            = (\nullset, u_3).\)
    \end{adhoctheorem*}

    Let \(S_3 = \pname{unify}(S_2(\tau_1), \tau_2 \to u_4)
              = \pname{unify}((\pint \to u_2) \to u_2, (u_3 \to u_3) \to u_4)
              = S_2' S_1'
              = [ \pint / u_2
                , \pint / u_3
                , \pint / u_4
                ]\),
    with subvariables obtained from the subcomputation:
    \begin{adhoctheorem*}{Computation}[Unify]
      Let \(S_1' = \pname{unify}(\pint \to u_2, u_3 \to u_3)
                 = S_2'' S_1''
                 = [ \pint / u_2
                   , \pint / u_3
                   ]\),
      with subvariables obtained from:
      \begin{adhoctheorem*}{Computation}[Unify]
        Let \(S_1'' = \pname{unify}(\pint, u_3)
                    = [\pint / u_3]\).

        Let \(S_2'' = \pname{unify}(S_1'' u_2, S_1'' u_3)
                    = \pname{unify}(u_2, \pint)
                    = [\pint / u_2]\).
      \end{adhoctheorem*}

      Let \(S_2' = \pname{unify}(S_1' u_2, S_1' u_4)
                 = \pname{unify}(\pint, u_4)
                 = [\pint / u_4]\)
    \end{adhoctheorem*}
  \end{adhoctheorem*}
  
  This gives \(W(\nullset, (\plambda{f}{f 5})(\plambda{x}{x}))
                 = (S_3 S_2 S_1, S_3 u_4)
                 = ([ \pint / u_1
                    , \pint / u_2
                    , \pint / u_3
                    , \pint / u_4
                    ], \pint)\).
\end{example}

We can include the \(\pletin{\dots}{\dots}\) construct with the typing rule
\[
  \prftree{\Gamma, x \of \tau' \vdash e_1 \of \tau'}
          {\Gamma, x \of \tau' \vdash e_2 \of \tau}
          {\Gamma \vdash \pletin{x = e_1}{e_2} \of \tau}
\]
This gives the constraint rule
\[
  \bbrackets{\Gamma \vdash \pletin{x = e_1}{e_2} \of \tau}
    = \exists \tau'. \bbrackets{\Gamma, x \of \tau' \vdash e_1 \of \tau'} \land
                     \bbrackets{\Gamma, x \of \tau' \vdash e_2 \of \tau}.
\]
This gives the type inference case:
\begin{minted}[escapeinside=||]{haskell}
w(te, e) = case e of
  (let |\(x = e_1\)| in |\(e_2\)|) -> let (|\(S_1\)|, |\(\tau_1\)|) = w(te |\(\union\)| {x:u}, |\(e_1\)|)
                           |\(\,S_2\)|      = unify (|\(S_1\)(u)|, |\(\tau_1\)|)
                           (|\(S_3\)|, |\(\tau_2\)|) = w(|\(S_2\)||\(S_1\)|(te) |\(\union\)| {x:|\(\tau_1\)|}, |\(e_2\)|)
                        in (|\(S_3 S_2 S_1\)|, |\(\tau_2\)|)
\end{minted}








\end{document}
