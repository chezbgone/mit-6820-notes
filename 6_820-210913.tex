\documentclass[class=scrartcl]{standalone}
\usepackage{chez}

\begin{document}
\section{Lambda Calculus}
Lambda calculus gives us a way to write and apply functions without
needing to give them names.

The heart of \(\lambda\)-calculus are expressions.
They have the grammar
\[
  E = x
    \mid \lambda x . E
    \mid E E.
\]
The \(x\) is a variable.
The more interesting terms are
\begin{description}
  \item[application] \(E_1 E_2\)
    where \(E_1\) is the function
    and \(E_2\) is the argument.
    Application is left-associative,
    so \(E_1 E_2 E_3 E_4\) means \((((E_1 E_2) E_3) E_4)\).
  \item[abstraction] \(\lambda x. E\)
    where the \(x\) is called the \vocab{bound variable}
    and \(E\) is the \vocab{lambda term}.
    The body of abstraction extends as far right as possible.
\end{description}

\(\lambda\)-calculus follows \vocab{lexical scoping},
which means that variables are bound based on the closest definition.

A \vocab{free variables} are the variables that
are not bound by a \(\lambda\) within the expression.
In particular, we have the following rules:
\begin{itemize}
  \item \(FV(x) = \set{x}\)
  \item \(FV(E_1 E_2) = FV(E_1) \union FV(E_2)\)
  \item \(FV(\lambda x. E) = FV(E) \setminus \set{x}\)
\end{itemize}
A \vocab{combinator} is a \(\lambda\)-expression with no free variables.

To do calculation, we use a rule called \vocab{\(\beta\)-reduction},
which is a reduction rule telling us how to apply functions:
\[
  (\lambda x. E)E_a \to E[E_a / x].
\]
The expression \(E[E_a / x]\) means
we substitute \(E_a\) for \(x\) in \(E\).
More formally:
\begin{itemize}
  \item If \(E\) is a variable \(y\),
        \[
          y[E_a / x] = \begin{cases*}
              E_a & if \(x \equiv y\) \\[-1ex]
            y & otherwise.
          \end{cases*}
        \]
  \item If \(E\) is an application \(E_1 E_2\),
        \[
          (E_1 E_2)[E_a / x] = (E_1[E_a / x]) (E_2[E_a / x]).
        \]
  \item If \(E\) is an abstraction \(\lambda y. E\),
        \[
          (\lambda y. E)[E_a / x] =
          \begin{cases*}
            \lambda y. E & if \(x \equiv y\) \\[-1ex]
            \lambda z. E[z / y][E_a / x] & otherwise, % chktex 1
          \end{cases*}
        \]
        where \(z \notin FV(E) \union FV(E_a) \union FV(x)\).
        Note that we need the \([z / y]\) step in the case
        that \(E\) has the variable \(y\) already,
        to avoid substitution capture.
\end{itemize}

The process of the substitution \([z / y]\) has a more general name:
\vocab{\(\alpha\)-substitution}:
\[
  \lambda x. E \to \lambda y.E[y / x],
\]
where \(y \notin FV(E)\).

There is another rule called \vocab{\(\eta\)-reduction}:
\[
  \lambda x. E x \to E,
\]
where \(x \notin FV(E)\).

We call the subterms that match the left sides of these rules \vocab{redexes}
(for reducible expression).
We say that a term is in \vocab{normal form} if they contain no redexs,
i.e.\ we cannot apply any more rules.

\begin{example} % cons, head, tail
  Consider the \(\lambda\)-expressions
  \begin{align*}
    C &\equiv \lambda x. \lambda y. \lambda f. f x y \\
    H &\equiv \lambda x. \lambda y. x \\
    T &\equiv \lambda x. \lambda y. y
  \end{align*}
  These expressions correspond to
    \textsf{cons},
    \textsf{head}, and
    \textsf{tail}.
  For instance,
  \begin{align*}
    (C a b) H &\to ((\lambda x. \lambda y. \lambda f. f x y) a b)
                   (\lambda x. \lambda y. x) \\
              &\to (\lambda f. f a b)
                   (\lambda x. \lambda y. x) \\
              &\to (\lambda x. \lambda y. x) a b \\
              &\to a \\
    (C a b) T &\to ((\lambda x. \lambda y. \lambda f. f x y) a b)
                   (\lambda x. \lambda y. y) \\
              &\to (\lambda f. f a b)
                   (\lambda x. \lambda y. y) \\
              &\to (\lambda x. \lambda y. y) a b \\
              &\to b.
  \end{align*}
  These combinators can be interpreted multiple ways.
  Another way of interpreting \(C\), \(H\), and \(T\) are as
    \textsf{pair},
    \textsf{first}, and
    \textsf{second}.
\end{example}

\subsection{Church encoding}
We can also encode the natural numbers with \(\lambda\)-expressions.
Let
\begin{align*}
  0 &\equiv \lambda f. \lambda x. x \\
  1 &\equiv \lambda f. \lambda x. f x \\
  2 &\equiv \lambda f. \lambda x. f (f x) \\
  3 &\equiv \lambda f. \lambda x. f (f (f x)) \\
    \MTFlushSpaceAbove
    &\vdotswithin{\equiv}
\end{align*}
where we interpret the natural number \(n\) as
applying a function \(f\) to some argument \(x\), \(n\) times.

\DeclareDocumentCommand{\succ}{}{\operatorname{\textsf{succ}}}
Using this interpretation, we can define \(\succ\), the successor function.
In particular, \(\succ\) is a function that takes in a natural \(n\)
and returns another numeral that applies the function \(f\), \(n + 1\) times.
We know that \(n f x\) applies the function \(n\) times so
we apply \(f\) to the result of this once more to get the desired expression.
\begin{align*}
  \succ &\equiv \lambda n. (\lambda f. \lambda x. f (n f x)) \\
        &\equiv \lambda n. \lambda f. \lambda x. f (n f x).
\end{align*}

\DeclareDocumentCommand{\lambdaplus}{}{\operatorname{\textsf{plus}}}
Since a natural \(n\) is encoded as applying a function \(n\) times,
we can define \(\lambdaplus\) as repeated application of \(\succ\):
\begin{align*}
  \lambdaplus &\equiv \lambda m. \lambda n. m \succ n \\
              &\equiv \lambda m. m \succ.
\end{align*}
where \(m \succ n\) can be thought of as applying \(\succ\),
\(m\) times to \(n\).
The second line is obtained from \(\eta\)-reduction
on \(\lambda n. m \succ n\).

\DeclareDocumentCommand{\lambdatimes}{}{\operatorname{\textsf{times}}}
We can similarly define \(\lambdatimes\) to be
\[
  \lambdatimes \equiv \lambda m. \lambda n. m (\lambdaplus n) 0.
\]


\DeclareDocumentCommand{\lambdatrue}{}{\operatorname{\textsf{true}}}
\DeclareDocumentCommand{\lambdafalse}{}{\operatorname{\textsf{false}}}
We can also represent booleans as:
\begin{align*}
  \lambdatrue &\equiv \lambda x. \lambda y. x \\
  \lambdafalse &\equiv \lambda x. \lambda y. y
\end{align*}

\DeclareDocumentCommand{\lambdacond}{}{\operatorname{\textsf{cond}}}
If we want to make choices based on a boolean,
like the \mintinline{c}{b?x:y} operator in C,
note that \(\lambdatrue\) and \(\lambdafalse\) select
their first and second arguments respectively.
Therefore, we have the analogous operator in \(\lambda\)-calculus
\[
  \lambdacond = \lambda b. \lambda x. \lambda y. b x y.
\]
Note that this \(\eta\)-reduces to the identity function,
so we can actually just apply the boolean
to two values we want to choose betwen.

\DeclareDocumentCommand{\lambdaiszero}{}{\operatorname{\textsf{isZero}}}
We can write a function \(\lambdaiszero\)
by applying \(\lambda x. \lambdafalse\) to \(\lambdatrue\) \(n\) times.
In particular, for any natural other than \(0\),
the result will be \(\lambdafalse\),
and otherwise it will be \(\lambdatrue\).
Written out,
\begin{align*}
  \lambdaiszero &\equiv \lambda n. n (\lambda x. \lambdafalse) \lambdatrue \\
                &\equiv \lambda n. n (\lambdatrue \lambdafalse) \lambdatrue.
\end{align*}







\end{document}
