\documentclass[class=scrartcl]{standalone}
\usepackage{chez}


\begin{document}
\NewDocumentCommand{\ulsf}{m}{\underline{\textsf{\smash[b]{#1}}}}

\section{Types for information flow}
Reference: Myers, A. C. ``JFlow: practical mostly-static information flow control''. In POPL `99

Normally when people talk about type systems in imperative programming,
where evaluation involves updating a store of values,
standard types place a restriction on the program store
and what operations you can use on the values.

Another common feature of imperative languages is public and public variables,
which restricts operations in contexts where they do not apply.
More generally, these are called information flow properties.
We say that code has \vocab{confidentiality} if private information
is not leaked to public values.
Conversely, we say that code has \vocab{integrity}
if public values do not have control over
the internal state of private values.

\begin{example}[Confidentiality]
  If we have a private class (in Python, e.g.) \mintinline{python}{Passwords}
  and a public class \mintinline{python}{Wikipedia},
  then the following code is not confidential:
  \begin{minted}[linenos=false]{python}
pws: Passwords = Passwords()
wiki: Wikipedia = Wikipedia()

wiki.add_entry(str(pws))
  \end{minted}
  However, not all information flow violations are as explicit.
  If we see the statement \mintinline{python}{wiki.write("YES")},
  we may think that it does not leak information,
  but in the right context, it actually does leak information:
  \begin{minted}[linenos=false]{python}
if len(pws.get_password()) == 10:
  wiki.write("YES")
  \end{minted}
  There are even more ambiguous situations:
  \begin{minted}[linenos=false]{python}
p = pws.get_password()
if q: wiki.write("YES")
  \end{minted}
  If \mintinline{python}{p} and \mintinline{python}{q} are
  related to each other, then information gets leaks.
\end{example}

\begin{example}[Integrity]
  This code does not have integrity:
  \begin{minted}{python}
class Passwords:
  change_password(self):
    self.password = wiki.entry[:10]
  \end{minted}
\end{example}

One may think that a solution would just to not have these
pieces of code interacting with each other at all,
but there are other use cases that have similar requirements.
For instance, a medical program may deal with many patients,
and we want to make sure there is no information flow between each
individual patient.
Another example is a browser, where we do not want information
from the saved passwords file to be leaked other than
a few specific situations in which the user allows it to.

More formally, if we have a program \(P\) that has both
public (\(L\)) and private (\(H\)) inputs and outputs,
Then for all \(L_i, H_i, H_i'\),
then \(F(L_i, H_i) = (L_o, H_o) \implies F(L_i, H_i') = (L_o, H_o')\),
i.e.\ changing the private input will not change the public output.
Note that we cannot determine this from a single execution of the program.
Because of this, it turns out our type based approach will work fairly well.

\subsection{Strategy}
We will first define a checkable property that implies
that information flow is secure.
Note that this may be stronger than what we actually need,
e.g.\ a program may be insecure in an unreachable branch,
but we will still reject the program.

This property will be defined with a \emph{dynamic} labeling scheme.
We will attach a label to every private value,
and then we will propagate it to other values that use it.
This turns a global property about executions with all possible inputs
to a local property.

\begin{example}
  If \(x\) has a private label, and we have \(z = a \cdot x\),
  then we make \(z\) also have a private label.
  Note that even if there is no information flow (like when \(a = 0\)),
  \(z\) will still have the label.

  It turns out that \(\textsf{NaN} \cdot 0 = \textsf{NaN}\),
  so it pays to be conservative in some of these cases.
\end{example}

We then will define a \emph{static} type system that allows us to
approximate the set of labels that values can have.

\subsection{Labels}
Before, our examples just had
one high confidentiality channel and one public channel.
In more complicated programs, there are often different private channels,
so we need a scheme that supports this.

We have principals, representing users or other identities,
such as groups or roles.
We then define a \emph{policy} to be
  a principal called the \emph{owner}, and a set
  containing principals called the \emph{readers}.
Policies will be of the form \textsf{Owner: (reader1, reader2, reader3)}.
Then, a label will be a set of policies.
The reason why we want owners is so that
permissions are able to change in the future
if an owner wants to update it.

Note that there is a partial ordering on these labels.
In particular, we write \(L_1 \leq L_2\)
if \(L_2\) is more restrictive than \(L_1\).
Note that this is slightly counterintuitive,
because \(L_2\) actually has fewer readers than \(L_1\).
This partial order forms a \vocab{lattice}, meaning
\begin{itemize}[nosep]
  \item we can take the least upper bound (\(\sqcup\)) of two elements,
  \item there exists a least fixed point
  \item there exists a minimum element, called bottom.
\end{itemize}

This partial order is similar to subtyping:
if a variable is trusted to handle a value with an \(L_2\) label,
then it is also trusted to handle a value with a label of \(L_1\).

\begin{example}
  \begin{gather*}
    \set{A: (B, C)} \leq \set{A: (B)} \\
    \set{A: (B, C), C: (B)} \leq \set{A: (B, C), C: (B)} \\
    \set{A: (B), C: (B)} \geq \set{A: (B, C)}
  \end{gather*}
  where the last one is because on the right side,
  \(C\) does not have any privacy concerns,
  i.e.\ it allows everything.
\end{example}

Now when we have some assignment \mintinline{text}|x{L2} := v{L1}|,
it is valid only if \mintinline{text}{L1 <= L2},
i.e.\ the variable being assigned to is more restrictive.
When we have a binary operation like \mintinline{text}|x{L1} + y{L2}|,
then the result has label \(L_1 \sqcup L_2\).
However, if we consider the following code:
\begin{minted}{Java}
int{A: everyone} a, b, c;
int{A: A} p;
c = 0;
if (p) { c = a + b; }
\end{minted}
Note that the information leaked into \mintinline{text}{c} is actually
given by the label of \mintinline{text}{p},
not just the join of the two labels.
The solution to this is to also consider the program counter PC.
In particular, we need to keep track of the information leaked
from knowing that the program is at a certain position.
Therefore, as we run the program, we also label the PC.

\subsection{Type system for information flow}
Our basic judgements will be of the form \(A \vdash E \of X\).
Here, the type environment \(A\) will carry a lot more information.
In particular, it will have information about the program counter.
Additionally, the type \(X\) will be a map containing several values:
\begin{itemize}[nosep]
  \item \(X[\ulsf{nv}]\) will be the label of \(E\)
        if it terminates normally.
  \item \(X[\ulsf{n}]\) will be the label that will be leaked if
        execution terminated after evaluating this expression.
\end{itemize}
We have the following rules:
\begin{gather*}
  \prfaxiom{A \vdash \textsf{literal} \of
                       X_\nullset[ \ulsf{n} \coloneqq A[\ulsf{pc}]
                                 , \ulsf{nv} \coloneqq A[\ulsf{pc} ]]} \\[0.5ex]
  \prfaxiom{A \vdash
            \pskip \of X_\nullset[ \ulsf{n} \coloneqq A[\ulsf{pc} ]]} \\[0.5ex]
  \prftree{A[v] = \angles{ \textrm{var}
                        \, [\textrm{final}]
                        \, \tau\set{L}
                        \, \textit{uid}} }
          {X = X_\nullset[ \ulsf{n} \coloneqq A[\ulsf{pc}]
                         , \ulsf{nv} \coloneqq L \sqcup A[\ulsf{pc}]
          ]}
          {A \vdash v \of X} \\
  \prftree{A \vdash E \of X}
          {A[v] = \angles{ \textrm{var}
                        \, \tau\set{L}
                        \, \textit{uid} }}
          {A \vdash X[\ulsf{nv}] \sqsubseteq L}
          {A \vdash v = E \of X} \\
  \prftree{A \vdash E \of X_E}
          {A[\ulsf{pc} \coloneqq X_E[\ulsf{nv}]] \vdash S_1 \of X_1}
          {A[\ulsf{pc} \coloneqq X_E[\ulsf{nv}]] \vdash S_2 \of X_2}
          {X = X_E[\ulsf{n} \coloneqq \underline{\nullset}]
               \oplus X_1 \oplus X_2}
          {A \vdash \pifthenelse{E}{S_1}{S_2} \of X} \\
  \prftree{A \vdash S_1 \of X_1}
          {\operatorname{extend}(A, S_1)[\ulsf{pc} \coloneqq X_1[\ulsf{n}]]
             \vdash S_2 \of X_2}
          {X = X_1[\ulsf{n} \coloneqq \underline{\nullset}] \oplus X_2}
          {A \vdash S_1; S_2 \of X}
\end{gather*}
\TODO[]
The first two rules are:
\begin{itemize}[nosep]
  \item If the expression is a literal, then the only information leaked
        is the program counter at that point.
  \item If the expression is a literal, then the only information leaked
        is the program counter at that point.
\end{itemize}







\end{document}
