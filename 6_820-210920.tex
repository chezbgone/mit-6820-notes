\documentclass[class=scrartcl]{standalone}
\usepackage{chez}
\usepackage{minted}

\begin{document}
% added to recursion in prev day

\section{Big Step Semantics}
Big step operational semantics models the execution in an abstract machine.
\begin{itemize}
  \item There are \vocab{judgements},
        written \(\angles{\text{configuration}} \to \text{result}\),
        describing how a program configuration is evaluated into a result.
        The configuration typically includes the program as well as any state.
  \item There are \vocab{inference rules},
        which describe how to derive judgements for an arbitrary program.
        These are sometimes called deriviation rules or evaluation rules.
\end{itemize}

\begin{example}
  Call by name evaluation order is described by:
  \[
    \prfaxiom{\angles{x} \to x} \qquad
    \prfaxiom{\angles{\lambda x. e} \to \lambda x. e} \qquad
    \prftree{\angles{e_1} \to \lambda x. e_1'}
            {\angles{e_1'[e_2 / x]} \to e_3}
            {\angles{\lambda x. e} \to \lambda x. e} \qquad
    \prftree{\angles{e_1} \to x}
            {\angles{e_1 e_2} \to x e_2} \qquad
    \prftree{\angles{e_1} \to e_a e_b}
            {\angles{e_1 e_2} \to e_a e_b e_2}
  \]
  Call by value evaluation order is described by:
  \begin{gather*}
    \prfaxiom{\angles{x} \to x} \qquad
    \prfaxiom{\angles{\lambda x. e} \to \lambda x. e} \qquad
    \prftree{\angles{e_1} \to \lambda x. e_1'}
            {\angles{e_2 \to e_3} \to e_3}
            {\angles{e_1'[e_2 / x]} \to e_2'}
            {\angles{e_1 e_2} \to e_3}\\
    \prftree{\angles{e_1} \to x}
            {\angles{e_2} \to e_2'}
            {\angles{e_1 e_2} \to x e_2'} \qquad
    \prftree{\angles{e_1} \to e_a e_b}
            {\angles{e_2} \to e_2'}
            {\angles{e_1 e_2} \to e_a e_b e_2'}
  \end{gather*}
\end{example}
Often, we drop the angle brackets for convenience.

\begin{example}
  To prove \((\lambda x. x)((\lambda y.y) z) \to z\), we have the proof tree
  \[
    \prftree{\prfaxiom{\lambda x. x \to \lambda x. x}}
            {\prftree{\prfaxiom{\lambda y. y \to \lambda y. y}}
                     {\prfaxiom{y[z / y] \to z}}
                     {(\lambda y. y) z \to z}
            }
            {(\lambda x. x)((\lambda y.y) z) \to z}
  \]
\end{example}



\section{Coq}
Before we work with big-step semantics,
we will first introduce a tool we will use.
One good thing about Coq is that it tries to rely on as little as possible.

In Coq, we can introduce definitions and theorem,
and we prove them by applying steps called \vocab{tactics}.

For instance, we can define the natural numbers:
\begin{minted}{coq}
Inductive nat := O | S (n : nat).
Fixpoint plus (n m : nat) : nat :=
  match n with
  | O => m
  | S n' => S (plus n' m)
  end.
\end{minted}

Here \mintinline{coq}{O} represents zero, and
\mintinline{coq}{S} represents successor.

To define a theorem and its proof, we can write:
\begin{minted}{coq}
Lemma O_plus : forall n, plus O n = n.
Proof.
  (* sequence of tactics *)
Qed.
\end{minted}

Here, the word \icoq{Lemma} can be replaced with any of
\icoq{Theorem},
\icoq{Remark},
\icoq{Corollary},
\icoq{Fact}, or
\icoq{Proposition}.

Proving theorems in Coq is a very interactive process,
so it is impossible to recreate on paper.
It is strongly recommended to try running the Coq code,
as just knowing the tactics is only a small part of the tool.

When running Coq code interactively,
we can check the status of a proof
by running the code up to some point in the proof.
There is a horizontal line,
above which is a list of hypotheses:
objects we know exist or statements we know are true.
Below the line is the goal we want to prove.


\subsection{Tactics}
In order to complete proofs,
we apply tactics that use the hypotheses to resolve our goal.

\begin{itemize}
  \item \icoq{reflexivity} can be used if our goal
        is resolved by normalizing terms,
        e.g.\ \icoq{plus O (S O) = S O}.
  \item \icoq{induction x} can be used to prove a goal
        by induction on a quantified variable \icoq{x}.
        Here, \icoq{x} is a inductively defined data type.
        Note that all variables quantified before \icoq{x}
        will be fixed throughout the induction.
        More on this in the next section.
  \item \icoq{simpl} will generally simplify your goal.
        Typically, it will include doing some \(\beta\)-reduction.
  \item \icoq{rewrite H} will rewrite the goal using a theorem/hypothesis
        \icoq{H : a = b}, by replacing \icoq{a} with \icoq{b}. % chktex 26
        Similarly, \icoq{rewrite <- H} will replace \icoq{b} with \icoq{a}.
  \item \icoq{intro} moves a quantified variable or hypothesis above the line.
        \icoq{intros} will do this as many as possible.
  \item \icoq{apply thm} will apply a theorem,
        reducing the goal to a subgoals
        for each of the theorem's hypotheses.
        \icoq{H : a = b}, by replacing \icoq{a} with \icoq{b}. % chktex 26
  \item \icoq{assumption} will solve a goal that matches a hypothesis.
  \item \icoq{destruct E} will do case analysis on some term \icoq{E}.
\end{itemize}




\end{document}
